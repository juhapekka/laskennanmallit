\documentclass[12pt,a4paper]{article}
% !TEX program = xelatex
\usepackage[utf8]{inputenc}
\usepackage[T1]{fontenc}
\usepackage[finnish]{babel}
\usepackage[utf8]{inputenc}
\usepackage{graphicx}
\usepackage{titling}
\usepackage{titlesec}
\usepackage{booktabs}
\usepackage{fancyhdr}
\usepackage{lipsum}
\usepackage{comment, mdframed}
\usepackage{enumitem}
\usepackage{xcolor}
\usepackage{longtable}
%\usepackage{cite}
\usepackage{pgfgantt}
\usepackage{amsmath, amssymb}
\usepackage{tikz}
\usepackage[margin=1in]{geometry}
\usepackage[backend=biber, style=numeric]{biblatex}
%\usepackage{hyperref}
\usepackage{bookmark}
\usepackage{enumitem}
\usepackage{amsmath}
\usepackage{listings}
\lstset{language=Python, basicstyle=\ttfamily\small, breaklines=true,columns=fullflexible}
\lstset{escapeinside={(*@}{@*)}}
\usepackage{fontspec}
\setmainfont{Arial}
\newfontfamily\stylishfont{Noteworthy}
%\newfontfamily\stylishfont{Zapfino}
%\addbibresource{references.bib}
\usetikzlibrary{calc}
\usepackage{xcolor}

\lstdefinestyle{pidstyle}{
    basicstyle=\ttfamily\footnotesize,
    breaklines=true,
    escapechar=\#, % Define escape character for inline LaTeX commands
    linewidth=\textwidth,
    basicstyle=\ttfamily\scriptsize
}

\renewcommand{\maketitle}{%
  \begin{leftmark}
    \vspace*{\baselineskip} % Add a bit of vertical space

%    \includegraphics[width=4cm]{example-image-a} % Add an image before the title. you will need to replace the image path with your own

%    \vspace{0.5cm} % Add vertical space before title

    \textbf{\fontsize{18}{36}\selectfont \thetitle} % Font Size and Bold Title

     \vspace{0.05cm} % Add vertical space before subtitle
%    \textit{\Large \theauthor}  % Subtitle / Author
    \vspace{\baselineskip} % Add vertical space after subtitle
     \rule{\textwidth}{0.4pt} % Add a horizontal line

   \end{leftmark}
%    \thispagestyle{empty} % Prevent header/footer on the title page
}


% Section Formatting
\titleformat{\section}
  {\normalfont\fontsize{18}{22}\bfseries} % Font and style
  {\thesection}         % Section number
  {1em}                   % Horizontal space after section number
  {}                     % Code before the section name
  []                     % Code after the section name

\titleformat{\subsection}
  {\normalfont\fontsize{14}{18}\bfseries} % Font and style
  {\thesubsection}         % Subsection number
  {1em}                   % Horizontal space after subsection number
  {}                     % Code before the subsection name
  []                     % Code after the subsection name

\setlength{\parindent}{0pt}

\title{Computing platforms (Spring 2025)\newline
week 6}
\author{Juha-Pekka Heikkilä}



\pagestyle{fancy}
\fancyhf{}

\renewcommand{\headrulewidth}{0pt}

\newcommand{\footerline}{\makebox[\textwidth]{\hrulefill}}

\newcommand{\footercontent}{%
    \begin{tabular}{@{}l@{}}
        \footerline \\
        \leftmark \hfill \rlap{\thepage}
    \end{tabular}
}

\fancyfoot[C]{\footercontent}


\newcommand{\exercise}[1]{
    \section*{Tehtävä #1}
    \markboth{Tehtävä #1}{}
}

\addtolength{\hoffset}{-1.75cm}
\addtolength{\textwidth}{3.5cm}
%\addtolength{\voffset}{-3cm}
%\addtolength{\textheight}{6cm}
%\setlength{\parindent}{0pt}



% (a), (b), (c)
\newlist{kohta}{enumerate}{1}
\setlist[kohta,1]{
  label=\textbf{\makebox[1cm][l]{\Huge\text{(\stylishfont\alph*)}}},
  leftmargin=!,
  labelindent=0pt
}

% (1), (2), (3)
\newlist{alakohta}{enumerate}{1}
\setlist[alakohta,1]{
  label=\textbf{\makebox[1cm][l]{\Large\text{(\arabic*)}}},
  leftmargin=!,
  labelindent=0pt
}

% termi: selitys
\newlist{kuvaus}{description}{1}
\setlist[kuvaus]{%
  font=\bfseries\stylishfont,
  labelsep=0.5cm,
  leftmargin=2.5cm,
  style=nextline
}

\newcommand{\korostus}[2][yellow]{\colorbox{#1}{\strut #2}}
%\korostus{Yksi kirjoittaja on jo sisällä}
%\korostus[red]{Lukijan täytyy odottaa jos kirjoittajia on paikalla}
%\korostus[orange]{Tämä osa ei ole suoritettavissa}


\newcommand{\evalslantti}[4][-12]{%
%  \left. #2 \,\right|% ei indeksejä tähän
  \mkern-10mu\raisebox{0pt}[0pt][0pt]{\rotatebox{#1}{$\Big|$}}% vinoviiva päälle
  \mkern3mu{}_{\!#3}^{\!#4}% arvot viivan oikealle puolelle
}


\newcommand{\evalraise}{1.2ex}
\newcommand{\evallow}{1.2ex}

% vino eval-viiva, arvot oikealla (oletus: -12)
% \evalslant[asteet]{lauseke}{ala}{yla}
\newcommand{\evalslant}[4][-12]{%
  \left. #2 \,\right.%
  \mkern-10mu\raisebox{0pt}[0pt][0pt]{\rotatebox{#1}{$\Big|$}}%
  \mkern2mu{}^{\raisebox{\evalraise}{$\scriptstyle #4$}}_{\raisebox{-\evallow}{$\scriptstyle #3$}}%
}



% vino eval-viiva ENNEN lauseketta
% \evalslantpre[asteet]{lauseke}{ala}{yla}
\newcommand{\evalslantpre}[4][-12]{%
  % viiva ja rajat
  \raisebox{0pt}[0pt][0pt]{\rotatebox{#1}{$\Big|$}}%
  \mkern2mu{}^{\raisebox{\evalraise}{$\scriptstyle #4$}}_{\raisebox{-\evallow}{$\scriptstyle #3$}}%
  % itse lauseke
  \mkern4mu\left. #2 \right.%
}


\DeclareMathOperator{\Var}{Var}
\DeclareMathOperator{\Cov}{Cov}
\DeclareMathOperator{\Corr}{Corr}
\usepackage{tikz}
\usetikzlibrary{automata, positioning, arrows.meta}
\usepackage{amssymb,amsmath,graphicx,color}
\usepackage{forest}

\newcommand{\set}[1]{\left\{\,#1\,\right\}}
\newcommand{\abs}[1]{\lvert#1\rvert}

\newcommand{\N}{\mathbb{N}}
\newcommand{\Pot}{{\cal P}}
\newcommand{\NPee}{\mathrm{NP}}
\newcommand{\code}[1]{\left\langle\,#1\,\right\rangle}

\newcommand{\rma}{\mathrm{a}}
\newcommand{\rmb}{\mathrm{b}}
\newcommand{\rmc}{\mathrm{c}}


\newcommand{\Pee}{\mathrm{P}}


\title{TKT20005 Laskennan mallit Viikko7}
\date{}

\begin{document}

\maketitle

\exercise{1 Esimerkkejä $\NPee$-täydellisistä ongelmista.} Tässä tehtävässä tutustutaan muutamaan esimerkkiin $\NPee$-täydellisistä ongelmista ja opetellaan määrittelemään ongelmia formaaleina kielinä.
  
{\em Ositus} (partition),
{\em laatikonpakkaus} (bin packing) ja {\em verkon värittäminen} (graph coloring) ovat kolme tunnettua
$\NPee$-täydellistä ongelmaa.
Etsi nämä ongelmat oppikirjoista tai verkosta ja selitä,
mistä niissä on kysymys.
Sopiva vastaus on lyhyt sanallinen kuvaus ongelmalle ja sen
täsmällinen esitys formaalina kielenä.

\bigskip

\begin{alakohta}

\item \textbf{Ositus (Partition) https://en.wikipedia.org/wiki/Partition\_problem}\\
Annettuna monijoukko kokonaislukuja $(a_1,\ldots,a_n)$. Onko se jaettavissa
kahteen osaan, joiden summat ovat samat?\\[2pt]
\emph{Kielenä:}
\[
\textsc{PARTITION}
=\Bigl\{\,\langle a_1,\ldots,a_n\rangle\ \Bigm|\ \exists I\subseteq\{1,\ldots,n\}:\ 
\sum_{i\in I} a_i \;=\; \sum_{i\notin I} a_i \,\Bigr\}.
\]

\item \textbf{Laatikonpakkaus (Bin Packing) https://en.wikipedia.org/wiki/Bin\_packing\_problem}\\
Annettuna esineiden koot $s_1,\ldots,s_n$, laatikon kapasiteetti $B$ ja
laatikkomäärä $k$. Voidaanko kaikki esineet jakaa enintään $k$ laatikkoon siten, että
kunkin laatikon kokonaiskoko $\le B$?\\[2pt]
\emph{Kielenä:}
\[
\textsc{BINPACK}
=\Bigl\{\,\langle s_1,\ldots,s_n,B,k\rangle\ \Bigm|\ 
\exists f:\{1,\ldots,n\}\to\{1,\ldots,k\}\ \text{siten, että}\ 
\forall j\in\{1,\ldots,k\}:\ \sum_{i:\,f(i)=j} s_i \le B \,\Bigr\}.
\]

\item \textbf{Verkon värittäminen (3-väritys) https://en.wikipedia.org/wiki/Graph\_coloring}\\
Annettuna suunnaton verkko $G=(V,E)$. Onko solmut väritettävissä
kolmella värillä niin, että mikään kaaren päätepistepari ei saa samaa väriä?\\[2pt]
\emph{Kielenä:}
\[
\textsc{3COLOR}
=\Bigl\{\,\langle G\rangle\ \Bigm|\ \exists c:V\to\{1,2,3\}\ \text{siten, että}\ 
\forall \{u,v\}\in E:\ c(u)\neq c(v) \,\Bigr\}.
\]

\end{alakohta}





\pagebreak
\exercise{2 Luokan $\NPee$ ongelmat.} Tässä tehtävässä harjoitellaan, kuinka todistetaan, että jokin ongelma kuuluu luokkaan $\NPee$.
  
{\em Repunpakkausongelmassa} (knapsack problem)
on annettu joukko esineitä $I$, jokaiselle esineelle $i\in I$
sen arvo $v_i\in\N$ ja paino $w_i\in\N$, repun painoraja $W\in\N$
ja tavoiteltu kokonaisarvo $V\in\N$.
Tehtävänä on ratkaista, onko mahdollista valita joukko $J\subseteq I$
esineitä reppuun pakattavaksi niin, että niiden yhteenlaskettu
arvo on ainakin $V$, mutta painoraja $W$ ei ylity.
Formaalina kielenä tämän voi määritellä
\[\mathit{KNAPSACK}=\{\,\code{I,v,w,V,W}\mid
\mbox{jollain $J\subseteq I$ pätee $\sum_{j\in J}w_j\leq W$ ja %
$\sum_{j\in J}v_j\geq V$}\,\}.\]
\begin{enumerate}
\item
Osoita, että kieli {\it KNAPSACK} kuuluu luokkaan $\NPee$.
Sopiva vastauksen tarkkuustaso on joko polynomisessa ajassa toimivan
epädeterministisen Turingin koneen tai polynomisen tarkastajan
kuvaus samaan tapaan kuin kurssimateriaalissa esim.\ 
$\mathit{CLIQUE}$-ongelmalle (luennot s.\ 259; Sipser s.\ 296).
\item {\bf Vapaaehtoinen lisätehtävä pohdittavaksi:} 
Repunpakkausongelmalle tunnetaan ajassa $O(nW)$ toimiva algoritmi,
missä $n=\abs{I}$ on esineiden lukumäärä.
Minkä takia
tästä ei seuraa $\mathit{KNAPSACK}\in\Pee$?
\end{enumerate}
{\em Lisätieto:} {\it KNAPSACK} on tunnettu NP-täydellinen ongelma.

\bigskip



https://en.wikipedia.org/wiki/Knapsack\_problem

\paragraph{} Given a set of items, each with a weight and a value, determine which items to include in the collection so that the total weight is less than or equal to a given limit and the total value is as large as possible.

\bigskip

\begin{alakohta}
\item \textbf{KNAPSACK $\in \NPee$.}
Käytetään binäärikoodausta; instanssi
$x=\langle (v_1,w_1),\ldots,(v_n,w_n),V,W\rangle$

\emph{Sertifikaatti:} bittijono $b\in\{0,1\}^n$, jossa $b_i=1$
iff esine $i$ valitaan.

\emph{Verifioija:} laskee
\[
S_w=\sum_{i=1}^n b_i w_i,\qquad S_v=\sum_{i=1}^n b_i v_i
\]
ja \textbf{hyväksyy} iff $S_w\le W$ ja $S_v\ge V$.

Koska summat ja vertailut binääriluvuilla vievät aikaa
$\mathrm{poly}(|x|)$, tarkastus on polynominen. 

Siis 
$\mathit{KNAPSACK}\in\NPee$
\end{alakohta}






\pagebreak
\exercise{3 $\NPee$-täydellisyys ja $\Pee$ vs $\NPee$ -ongelma.} Tässä tehtävässä tarkastellaan, miten $\NPee$-täydellisyys ja $\Pee$ vs $\NPee$ -ongelma liittyvät toisiinsa.
  
Kuten muistetaan, Hamiltonin polku -ongelma {\it HAMPATH} on tunnettu
$\NPee$-täydellinen ongelma.
\begin{enumerate}
\item
Oletetaan, että joku löytää polynomisessa ajassa toimivan
ratkaisualgoritmin Hamiltonin polku -ongelmalle.
Mitä tästä seuraa $\Pee$ vs.\ $\NPee$ -ongelmalle?
\item
Entä jos joku todistaa, että Hamiltonin polku -ongelmaa ei ole mahdollista
ratkaista polynomisessa ajassa?
Mitä tästä seuraa $\Pee$ vs.\ $\NPee$ -ongelmalle?
\item Oletetaan, että joku todistaa, että $\Pee=\NPee$. Mitä tästä seuraa Hamiltonin polku \mbox{-ongelmalle?}
\item Entä jos joku todistaa, että $\Pee\neq\NPee$? Mitä tästä seuraa Hamiltonin polku -ongelmalle?
\end{enumerate}
Kaikissa kohdassa perustele vastauksesi täsmällisesti
lähtien luokan $\NPee$ ja $\NPee$-täydellisyyden määritelmistä ja tiedosta,
että {\it HAMPATH} on $\NPee$-täydellinen.
Siis tarkoitus ei ole käyttää suoraan esim.\ lausetta~5.27,
vaan esittää sen taustalla oleva päättely tähän
tilanteeseen sovellettuna.

\bigskip

\begin{alakohta}
\item % (1)
\textbf{Jos $\mathit{HAMPATH}$ ratkaistaan polynomiajassa, niin $P=NP$}\\
Koska $\mathit{HAMPATH}$ on NP-täydellinen, pätee
(i) $\mathit{HAMPATH}\in NP$ ja
(ii) jokaiselle $L\in NP$ on polynominen monikoinen reduktio $L \le_m^p \mathit{HAMPATH}$.
Oletuksesta $\mathit{HAMPATH}\in P$ ja (ii):sta seuraa, että \emph{kaikki} $L\in NP$ ovat
myös luokassa $P$ (komponeeraamalla reduktion ja $\mathit{HAMPATH}$:n poli-algoritmin).
Siis $NP\subseteq P$. Toisaalta aina $P\subseteq NP$, joten $P=NP$.

\item % (2)
\textbf{Jos $\mathit{HAMPATH}$:lle ei ole polynomista algoritmia, niin $P\neq NP$.}\\
Koska $\mathit{HAMPATH}\in NP$ (NP-täydellisyyden ehto (i)), väite “ei ole polynomista
algoritmia” tarkoittaa $\mathit{HAMPATH}\notin P$. Tällöin on olemassa NP-kieli, joka ei kuulu
$P$:hen, joten $P\neq NP$ (koska aina $P\subseteq NP$).

\item % (3)
\textbf{Jos $P=NP$, niin $\mathit{HAMPATH}\in P$.}\\
NP-täydellisyyden ehto (i): $\mathit{HAMPATH}\in NP$. Yhtälöstä $P=NP$ seuraa
suoraan, että $\mathit{HAMPATH}\in P$.

\item % (4)
\textbf{Jos $P\neq NP$, niin $\mathit{HAMPATH}\notin P$.}\\
Oletetaan vastaoletuksena $\mathit{HAMPATH}\in P$.
Koska ehto (ii) sanoo, että jokainen $L\in NP$ redusoituu polynomiajassa
$\mathit{HAMPATH}$:iin, tästä seuraisi $NP\subseteq P$.
Yhdessä triviaalin $P\subseteq NP$ kanssa saataisiin $P=NP$, mikä
on ristiriidassa oletuksen $P\neq NP$ kanssa.
Siis $\mathit{HAMPATH}\notin P$.
\end{alakohta}


\pagebreak
\exercise{4 Polynomisen reduktion transitiivisuus.} Luentomateriaalissa lauseen~5.29 todistuksessa käytettiin tässä tehtävässä todistettavaa lemmaa: jos $A\leq_P B$ ja $B\leq_P C$, niin $A\leq_P C$. Tämä tehtävä siis täydentää luentomateriaalia tältä osin.
  
Oletetaan, että funktiot
$f\colon\Sigma^\ast\rightarrow\Sigma^\ast$ ja
$g\colon\Sigma^\ast\rightarrow\Sigma^\ast$ 
voidaan laskea polynomisessa ajassa.
Osoita, että myös yhdistetty funktio
$h\colon\Sigma^\ast\rightarrow\Sigma^\ast$, missä $h(x)=g(f(x))$, 
voidaan laskea polynomisessa ajassa.
Päättele tästä edelleen, että jos
$A\leq_P B$ ja $B\leq_P C$, niin $A\leq_P C$.

{\em Huom.}
Muista, että aikavaativuus määritellään
syötteen koon suhteen.
Ota huomioon, minkä kokoisilla syötteillä funktion
$g$ arvoa lasketaan.

\bigskip




{\bf lemma}\\
olkoon $f,g:\Sigma^*\to\Sigma^*$ polynomiajassa laskettavia.
Tällöin myös $h(x)=g(f(x))$ on polynomiajassa laskettava.\\

\emph{Todistus.}
On olemassa polynomit $p,s$ siten, että $f$ voidaan laskea ajassa $p(n)$
syötteen pituudella $n$, ja $g$ voidaan laskea ajassa $s(m)$ syötteen
pituudella $m$.\\

Kun lasketaan $h(x)$ syötteellä $x$, tehdään ensin $f(x)$ ajassa $p(|x|)$.
Samalla saadaan $\lvert f(x)\rvert \le p(|x|)$, koska yhden merkin kirjoituskin
vie yhden askeleen.

Seuraavaksi ajetaan $g$ syötteellä $f(x)$; tämän aika on
$s(\lvert f(x)\rvert)\le s(p(|x|))$.
Kokonaisaika on siis
\[
T_h(|x|) \;\le\; p(|x|)\;+\;s(p(|x|))
\]
mikä on polynomi koska polynomin sijoitus polynomiin on polynomi \\

Siis $h$ on polynomiajassa laskettava. \quad \(\square\)

\bigskip
\textbf{Seuraus.}
Jos $A\le_P B$ funktiolla $f$ ja $B\le_P C$ funktiolla $g$, määritä $h=g\circ f$.
Lemmasta $h$ on polynominen\\
 Lisäksi kaikilla $x$:
\[
x\in A \iff f(x)\in B \iff g(f(x))\in C \iff h(x)\in C
\]
Siksi $A\le_P C$ \quad \(\square\)

\medskip
\emph{Huom} \Rightarrow\, aikavaativuus määritellään syötteen koon suhteen:\\

g:n aika riippuu sen oman syötteen pituudesta $|f(x)|$, joka on
polynomisesti sidottu $|x|$:ään, koska $f$ on polynominen ja ei voi tulostaa
enempää symbooleja kuin mitä sen laskenta askeleita on

\end{document}