\documentclass[12pt,a4paper]{article}
% !TEX program = xelatex
\usepackage[utf8]{inputenc}
\usepackage[T1]{fontenc}
\usepackage[finnish]{babel}
\usepackage[utf8]{inputenc}
\usepackage{graphicx}
\usepackage{titling}
\usepackage{titlesec}
\usepackage{booktabs}
\usepackage{fancyhdr}
\usepackage{lipsum}
\usepackage{comment, mdframed}
\usepackage{enumitem}
\usepackage{xcolor}
\usepackage{longtable}
%\usepackage{cite}
\usepackage{pgfgantt}
\usepackage{amsmath, amssymb}
\usepackage{tikz}
\usepackage[margin=1in]{geometry}
\usepackage[backend=biber, style=numeric]{biblatex}
%\usepackage{hyperref}
\usepackage{bookmark}
\usepackage{enumitem}
\usepackage{amsmath}
\usepackage{listings}
\lstset{language=Python, basicstyle=\ttfamily\small, breaklines=true,columns=fullflexible}
\lstset{escapeinside={(*@}{@*)}}
\usepackage{fontspec}
\setmainfont{Arial}
\newfontfamily\stylishfont{Noteworthy}
%\newfontfamily\stylishfont{Zapfino}
%\addbibresource{references.bib}
\usetikzlibrary{calc}
\usepackage{xcolor}

\lstdefinestyle{pidstyle}{
    basicstyle=\ttfamily\footnotesize,
    breaklines=true,
    escapechar=\#, % Define escape character for inline LaTeX commands
    linewidth=\textwidth,
    basicstyle=\ttfamily\scriptsize
}

\renewcommand{\maketitle}{%
  \begin{leftmark}
    \vspace*{\baselineskip} % Add a bit of vertical space

%    \includegraphics[width=4cm]{example-image-a} % Add an image before the title. you will need to replace the image path with your own

%    \vspace{0.5cm} % Add vertical space before title

    \textbf{\fontsize{18}{36}\selectfont \thetitle} % Font Size and Bold Title

     \vspace{0.05cm} % Add vertical space before subtitle
%    \textit{\Large \theauthor}  % Subtitle / Author
    \vspace{\baselineskip} % Add vertical space after subtitle
     \rule{\textwidth}{0.4pt} % Add a horizontal line

   \end{leftmark}
%    \thispagestyle{empty} % Prevent header/footer on the title page
}


% Section Formatting
\titleformat{\section}
  {\normalfont\fontsize{18}{22}\bfseries} % Font and style
  {\thesection}         % Section number
  {1em}                   % Horizontal space after section number
  {}                     % Code before the section name
  []                     % Code after the section name

\titleformat{\subsection}
  {\normalfont\fontsize{14}{18}\bfseries} % Font and style
  {\thesubsection}         % Subsection number
  {1em}                   % Horizontal space after subsection number
  {}                     % Code before the subsection name
  []                     % Code after the subsection name

\setlength{\parindent}{0pt}

\title{Computing platforms (Spring 2025)\newline
week 6}
\author{Juha-Pekka Heikkilä}



\pagestyle{fancy}
\fancyhf{}

\renewcommand{\headrulewidth}{0pt}

\newcommand{\footerline}{\makebox[\textwidth]{\hrulefill}}

\newcommand{\footercontent}{%
    \begin{tabular}{@{}l@{}}
        \footerline \\
        \leftmark \hfill \rlap{\thepage}
    \end{tabular}
}

\fancyfoot[C]{\footercontent}


\newcommand{\exercise}[1]{
    \section*{Tehtävä #1}
    \markboth{Tehtävä #1}{}
}

\addtolength{\hoffset}{-1.75cm}
\addtolength{\textwidth}{3.5cm}
%\addtolength{\voffset}{-3cm}
%\addtolength{\textheight}{6cm}
%\setlength{\parindent}{0pt}



% (a), (b), (c)
\newlist{kohta}{enumerate}{1}
\setlist[kohta,1]{
  label=\textbf{\makebox[1cm][l]{\Huge\text{(\stylishfont\alph*)}}},
  leftmargin=!,
  labelindent=0pt
}

% (1), (2), (3)
\newlist{alakohta}{enumerate}{1}
\setlist[alakohta,1]{
  label=\textbf{\makebox[1cm][l]{\Large\text{(\arabic*)}}},
  leftmargin=!,
  labelindent=0pt
}

% termi: selitys
\newlist{kuvaus}{description}{1}
\setlist[kuvaus]{%
  font=\bfseries\stylishfont,
  labelsep=0.5cm,
  leftmargin=2.5cm,
  style=nextline
}

\newcommand{\korostus}[2][yellow]{\colorbox{#1}{\strut #2}}
%\korostus{Yksi kirjoittaja on jo sisällä}
%\korostus[red]{Lukijan täytyy odottaa jos kirjoittajia on paikalla}
%\korostus[orange]{Tämä osa ei ole suoritettavissa}


\newcommand{\evalslantti}[4][-12]{%
%  \left. #2 \,\right|% ei indeksejä tähän
  \mkern-10mu\raisebox{0pt}[0pt][0pt]{\rotatebox{#1}{$\Big|$}}% vinoviiva päälle
  \mkern3mu{}_{\!#3}^{\!#4}% arvot viivan oikealle puolelle
}


\newcommand{\evalraise}{1.2ex}
\newcommand{\evallow}{1.2ex}

% vino eval-viiva, arvot oikealla (oletus: -12)
% \evalslant[asteet]{lauseke}{ala}{yla}
\newcommand{\evalslant}[4][-12]{%
  \left. #2 \,\right.%
  \mkern-10mu\raisebox{0pt}[0pt][0pt]{\rotatebox{#1}{$\Big|$}}%
  \mkern2mu{}^{\raisebox{\evalraise}{$\scriptstyle #4$}}_{\raisebox{-\evallow}{$\scriptstyle #3$}}%
}



% vino eval-viiva ENNEN lauseketta
% \evalslantpre[asteet]{lauseke}{ala}{yla}
\newcommand{\evalslantpre}[4][-12]{%
  % viiva ja rajat
  \raisebox{0pt}[0pt][0pt]{\rotatebox{#1}{$\Big|$}}%
  \mkern2mu{}^{\raisebox{\evalraise}{$\scriptstyle #4$}}_{\raisebox{-\evallow}{$\scriptstyle #3$}}%
  % itse lauseke
  \mkern4mu\left. #2 \right.%
}


\DeclareMathOperator{\Var}{Var}
\DeclareMathOperator{\Cov}{Cov}
\DeclareMathOperator{\Corr}{Corr}
\usepackage{tikz}
\usetikzlibrary{automata, positioning, arrows.meta}
\usepackage{amssymb,amsmath,graphicx,color}
\usepackage{forest}
\usepackage{siunitx}
\usepackage{amsthm}

\newtheorem{lause}{Lause}
\renewcommand{\proofname}{Todistus}
\renewcommand{\qedsymbol}{$\blacksquare$}

\newcommand{\set}[1]{\left\{\,#1\,\right\}}
\newcommand{\abs}[1]{\lvert#1\rvert}

\newcommand{\N}{\mathbb{N}}
\newcommand{\Pot}{{\cal P}}

\newcommand{\rma}{\mathrm{a}}
\newcommand{\rmb}{\mathrm{b}}
\newcommand{\rmc}{\mathrm{c}}
\newcommand{\code}[1]{\left\langle\,#1\,\right\rangle}
\newcommand{\Pee}{\mathrm{P}}



\title{TKT20005 Laskennan mallit Viikko6}
\date{}

\begin{document}

\maketitle

\exercise{1 Pysähtymisongelma.} Tässä tehtävässä pyritään ymmärtämään pysähtymisongelman määritelmä.\\

Mitä voit päätellä seuraavien väittämien totuusarvoista
kurssilla tähän mennessä esitettyjen määritelmien ja
tulosten perusteella?
(Vaihtoehdot ovat siis ''tosi'', ''epätosi'' ja
''annetut tiedot eivät riitä asian ratkaisemiseen''.)
Perustele.
\begin{alakohta}
\item
On olemassa algoritmi, joka ratkaisee mille tahansa
ohjelmalle $P$, millä syötteillä $P$ pysähtyy
ja millä ei.\\

{\bf epätosi}

(sivu 225)\\
Jos olisi algoritmi, joka kaikille ohjelmille ratkaisee, millä 
syötteillä $P$ pysähtyy ja millä ei, siitä tulisi
ratkaisija Universaalikielen pysähtymisongelmaan.\\

\item
On olemassa sellainen ohjelma $P$, että mikään algoritmi
ei pysty ratkaisemaan, millä syötteillä $P$ pysähtyy
ja millä ei.\\

{\bf tosi}

On olemassa tietty ohjelma $P$,
jonka pysähtymissyötteiden joukko ei ole ratkeava;
Universaali simulaattori $U$:
Ohjelma $P$, Syötteellä $y$: tulkitse $y$ koodina
$\langle M,w\rangle$ ja simuloi $M$:ää syötteellä $w$; 
jos $M$ pysähtyy, myös $P$ pysähtyy.\\

Tällöin $P$:n pysähtymisjoukko on $H$
joka ei ole ratkeava. Eli jonkin $P$:n kohdalla ei
ole algoritmia, joka kertoisi kaikille syötteille
pysähtyykö P\\


\item
Kaikille ohjelmille $P$ pätee, että mikään algoritmi
ei pysty ratkaisemaan, millä syötteillä $P$ pysähtyy
ja millä ei.\\

{\bf epätosi}

Väite on liian kattava.\\
Vastatapaus: ohjelma, joka pysähtyy kaikilla syötteillä
(pysähtymisjoukko $\Sigma^*$, siis ratkeava), tai ohjelma, 
joka ei pysähdy millään syötteellä
(pysähtymisjoukko $\emptyset$, eli taas ratkeava)

\end{alakohta}




\pagebreak
\exercise{2 Turing-ratkeamattomuus.} Tässä tehtävässä harjoitellaan, kuinka jokin kieli todistetaan ratkeamattomaksi.
    
Tarkastellaan kieltä
\[A=\set{\code{M,w,q}\mid
\mbox{$M$ on Turingin kone, joka syötteellä $w$ menee ainakin kerran
  tilaan $q$}}.\] Viime viikon tehtävässä 3 todistimme, että tämä kieli on Turing-tunnistettava.

Todista, että kieli $A$ ei ole ratkeava.\\

\bigskip

    
Tehdään vastaoletus että $A$ on ratkeava. Olkoon $R$ Turingin kone
joka päättää $A$\\

Käyttämällä apuna konetta $R$ hyväksymisongelma
\[
A_{\mathrm{TM}}=\{\langle M,w\rangle \mid \text{$M$ hyväksyy syötteen $w$}\}
\]
voitaisiin ratkaista Turingin koneell, joka syötteellä $y$ toimii:
\begin{alakohta}
\item Jos $y$ ei ole muotoa $\langle M,w\rangle$, hylkää

\item Poimi $M$:n koodista sen hyväksyvä tila $q_{\mathrm{acc}}$ ja muodosta
      $\langle M,w,q_{\mathrm{acc}}\rangle$

\item Simuloi konetta $R$ syötteellä $\langle M,w,q_{\mathrm{acc}}\rangle$
      \begin{itemize}
        \item Jos $R$ hyväksyi, niin hyväksy
        \item Jos $R$ hylkäsi, niin hylkää
      \end{itemize}
\end{alakohta}

$M$ hyväksyy $w$ täsmälleen silloin, kun sen laskennassa
syötteellä $w$ vieraillaan tilassa $q_{\mathrm{acc}}$ ainakin kerran;
eli täsmälleen silloin kun $\langle M,w,q_{\mathrm{acc}}\rangle\in A$.
Koska $R$ päättää $A$:n, yllä oleva kone päättää $A_{\mathrm{TM}}$.

Mutta $A_{\mathrm{TM}}$ ei ole ratkeava mistä seuraa ristiriita. Siis $A$ ei ole ratkeava. \qed








\pagebreak
\exercise{3 Vaativuusluokka $\Pee$} Tässä tehtävässä harjoitellaan, kuinka ongelma määritellään formaalina kielenä ja kuinka todistetaan, että jokin ongelma kuuluu vaativuusluokkaan $\Pee$.
  
  Tarkastellaan seuraavaa ongelmaa:

  \hspace{0.5cm}{\bf Annettu:} Lista kokonaislukuja $L$ ja kokonaisluku $k$.

  \hspace{0.5cm}{\bf Kysymys:} Esiintyykö $k$ listassa $L$?

  \begin{enumerate}
  \item Esitä yllä annettu ongelma formaalina kielenä.
  \item Todista, että määrittelemäsi formaali kieli kuuluu luokkaan $\Pee$.
  \end{enumerate}

  \bigskip

\begin{alakohta}
\item Esitä yllä annettu ongelma formaalina kielenä.\\

Käytetään aakkostoa $\Sigma=\{0,1,\#,-\}$. \\
Määritellään kokonaislukujen koodijoukko binäärisenä
\[
E \;=\; \{\,0\,\} \;\cup\; \{\, s\,1x \mid s\in\{\varepsilon,-\},\ x\in\{0,1\}^{*} \,\}
\]
eli luvun koodi on joko $0$ tai \{valinnainen miinus + $1$ + mielivaltainen bittijono\}\\
Silloin päätös ongelmaa esiintyykö $k$ listassa $L$? esittävä kieli on (\#\# erottaa 
"listan" ja "kohteen")
\[
L \;=\; \{\, u_1\#u_2\#\cdots\#u_m\#\#v \mid m\ge 0,\ \forall i:\ u_i\in E,\ v\in E,\ \exists i:\ u_i=v \,\}
\]


\pagebreak

\item Todista, että määrittelemäsi formaali kieli kuuluu luokkaan $\Pee$.\\

\[
\mathrm{B}
=\bigl\{\,\langle L,k\rangle \mid \text{$k$ esiintyy listassa $L$}\,\bigr\}
=\bigl\{\,u_1\#\cdots\#u_m\#\#v \mid \exists i:\ u_i=v\,\bigr\}
\]

Väite: $\mathrm{B}\in \mathbf{P}$\\


Laajennetaan nauha-aakkostoa työsymboleilla $r$ ja $t$
joilla merkitään jo verratut bitit. Syöte $x$, pituus $n$, on muodossa
$u_1\#\cdots\#u_m\#\#v$

\begin{enumerate}
  \item Muototarkistus.\\
   Kulje vasemmalta oikealle ja tarkista muoto
        $u_1\#\cdots\#u_m\#\#v$ ja että jokainen lohko on koodista
        $E=\{0\}\cup 1\{0,1\}^*$. Jos ei, hylkää
  \item \emph{Vertaa jokaista $u_i$:tä $v$:hen merkitsemällä.}
        Käy listalohkot $u_1,\ldots,u_m$ vasemmalta oikealle:
        \begin{enumerate}
          \item Palauta (tarvittaessa) $v$ ennalleen: korvaa kaikki
                $r, t$ $v$-osassa takaisin $0,1$:iksi
          \item Aloita $u_i$:n alusta. Toista:
                \begin{itemize}
                  \item Etsi $u_i$:stä vasemmalta lukien ensimmäinen merkitsemätön
                        symboli. Jos sellaista ei ole (eli tuli vastaan `\#`), tarkasta että
                        myös $v$:ssä ei ole merkitsemättömiä $0/1$-symboleja.
                        Jos ei ole, hyväksy
                  \item Lue löytynyt bitti $b\in\{0,1\}$; siirry $v$:n alkuun (skannaa `\#\#`)
                        ja etsi $v$:stä ensimmäinen merkitsemätön bitti.
                        Jos bitti on $b$, merkitse molemmat (kirjoita
                        $0\mapsto r$ tai $1\mapsto t$  sekä $u_i$:ssä että $v$:ssä) ja palaa
                        $u_i$:n alkuun jatkamaan
                \end{itemize}
        \end{enumerate}
  \item Jos kaikki lohkot käsiteltiin ilman hyväksyntää, hylkää.
\end{enumerate}

\textbf{Aika-analyysi.}
Kukin merkintä/vasta-merkinnän palautus voi vaatia pään siirtelyä
korkeintaan $O(n)$ solun yli (edestakaisin listan ja $v$:n välillä).
Yhtä lohkoa kohden merkitään enintään sen pituus + $|v|$ symboleja,
ja koko syötteen eri lohkoissa on yhteensä $O(n)$ symboleja.
Siten kokonaisaika on $O(n)\cdot O(n)=O(n^2)$.
Polynominen aika $\Rightarrow\ \mathrm{B}\in\mathbf{P}$

\end{alakohta}





\pagebreak
\exercise{4 Lisäharjoitusta ratkeamattomuustodistuksiin. }
Tavoitteena on laatia ohjelmantarkastustyökalu, johon voi ladata
tarkastettavaksi minkä tahansa Python-ohjelman.
Käyttäjä valitsee haluamansa syötteen annettavaksi
tarkastettavalle ohjelmalle.
Lisäksi käyttäjä valitsee ohjelmasta jonkin koodirivin.
Tarkastustyökalun tehtävänä on päättää,
suorittaisiko tarkastettava ohjelma
valitun koodirivin ainakin kerran, kun ohjelmalle annetaan
valittu syöte.\\

Onko tällainen tarkastustyökalu mahdollista toteuttaa niin,
että se varmasti toimii oikein millä tahansa ohjelmalla,
syötteellä ja ohjelmasta valitulla koodirivillä?
Perustele.\\



\bigskip

Olkoon {\sc Rivi} kieli mikä sisältää kolmikot $\langle P,x,\ell\rangle$,
kun Python-ohjelma $P$ syötteellä $x$ suorittaa koodirivin $\ell$ ainakin kerran.

\bigskip
\noindent\textbf{Väite} \\
Kieli {\sc Rivi} ei ole ratkeava; siis pyydettyä työkalua,
joka aina pysähtyy ja vastaa oikein kaikilla syötteillä, ei ole olemassa

\bigskip
\noindent\textbf{Perustelu} (hyväksymisongelmasta $A_{\rm TM}$)\\
Tiedetään että hyväksymisongelma
\[
A_{\rm TM}=\{\langle M,w\rangle \mid \text{$M$ hyväksyy syötteen $w$}\}
\]
on ratkeamaton.\\

Näytetään että jos {\sc Rivi} olisi ratkeava, myös $A_{\rm TM}$ olisi,
mikä on mahdotonta\\

Annetulle parille $\langle M,w\rangle$ rakennetaan Python-ohjelma $P_{M,w}$,
joka tekee:

\begin{quote}\ttfamily
simuloi $M$:ää syötteellä $w$;\\
\ \ $M$ \emph{hyväksyy}: suorita \underline{rivi L} ja pysähdy;\\
\ \ $M$ \emph{hylkää} tai ei pysähdy: älä koskaan suorita riviä L
\ (esim.\ jää silmukkaan)
\end{quote}

Tällöin
\[
\langle M,w\rangle\in A_{\rm TM}
\iff
\text{$P_{M,w}$ suorittaa rivin $L$ syötteellä $x$}
\]
(riippumatta vakiovalinnasta $x$, esim.\ $x=\varepsilon$)\\


Jos meillä olisi päät­tävä työkalu $C$ ongelmaan {\sc Rivi}, voisimme ratkaista
$A_{\rm TM}$:n algoritmilla: syötteellä $\langle M,w\rangle$ muodosta
$\langle P_{M,w},x,L\rangle$ ja palauta $C$:n vastaus. Tämä tekisi
$A_{\rm TM}$:stä ratkeavan eli siis ristiriita.\\

Siksi {\sc Rivi} ei ole ratkeava, eikä pyydettyä yleispätevää tarkastustyökalua
voi toteuttaa. \qed





\pagebreak
\exercise{5 Kieli, joka ei ole Turing-tunnistettava.}
  Todista, että kieli
\[B=\set{\code{M,w,q}\mid
\mbox{$M$ on Turingin kone, joka syötteellä $w$ {\it ei mene} kertaakaan
tilaan $q$}}.\]
ei ole Turing-tunnistettava.

({\em Vihje:} Tarkastele kielen $B$ komplementtia ja hyödynnä luentomonisteen lausetta 4.7 sekä tehtävän 2 ja viime viikon tehtävän 3 tulosta, että kieli
\[
A=\set{\code{M,w,q}\mid
\mbox{$M$ on Turingin kone, joka syötteellä $w$ menee ainakin kerran
  tilaan $q$}}
\]
on Turing-tunnistettava, mutta ei ratkeava.)

\bigskip

Olkoon
\[
A=\{\langle M,w,q\rangle \mid \text{$M$ syötteellä $w$ käy ainakin kerran tilassa $q$}\}
\]
ja
\[
B=\{\langle M,w,q\rangle \mid \text{$M$ syötteellä $w$ \emph{ei} käy kertaakaan tilassa $q$}\}
\]
Selvästi $B=\overline{A}$ (komplementti hyvinmuodostettujen kolmikkojen joukossa)\\


Viime viikolla todistettiin, että $A$ on Turing-tunnistettava, mutta ei ratkeava.
Oletetaan, että $B$ olisi Turing-tunnistettava. Tällöin sekä $A$ että
$\overline{A}$ olisivat tunnistettavia ja Lauseen~4.7
mukaan $A$ olisi ratkeava. Tämä on ristiriidassa edellä mainitun tuloksen kanssa.\\

Siis $B$ \emph{ei} ole Turing-tunnistettava. \(\square\)



\end{document}