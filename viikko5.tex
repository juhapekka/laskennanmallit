\documentclass[12pt,a4paper]{article}
% !TEX program = xelatex
\usepackage[utf8]{inputenc}
\usepackage[T1]{fontenc}
\usepackage[finnish]{babel}
\usepackage[utf8]{inputenc}
\usepackage{graphicx}
\usepackage{titling}
\usepackage{titlesec}
\usepackage{booktabs}
\usepackage{fancyhdr}
\usepackage{lipsum}
\usepackage{comment, mdframed}
\usepackage{enumitem}
\usepackage{xcolor}
\usepackage{longtable}
%\usepackage{cite}
\usepackage{pgfgantt}
\usepackage{amsmath, amssymb}
\usepackage{tikz}
\usepackage[margin=1in]{geometry}
\usepackage[backend=biber, style=numeric]{biblatex}
%\usepackage{hyperref}
\usepackage{bookmark}
\usepackage{enumitem}
\usepackage{amsmath}
\usepackage{listings}
\lstset{language=Python, basicstyle=\ttfamily\small, breaklines=true,columns=fullflexible}
\lstset{escapeinside={(*@}{@*)}}
\usepackage{fontspec}
\setmainfont{Arial}
\newfontfamily\stylishfont{Noteworthy}
%\newfontfamily\stylishfont{Zapfino}
%\addbibresource{references.bib}
\usetikzlibrary{calc}
\usepackage{xcolor}

\lstdefinestyle{pidstyle}{
    basicstyle=\ttfamily\footnotesize,
    breaklines=true,
    escapechar=\#, % Define escape character for inline LaTeX commands
    linewidth=\textwidth,
    basicstyle=\ttfamily\scriptsize
}

\renewcommand{\maketitle}{%
  \begin{leftmark}
    \vspace*{\baselineskip} % Add a bit of vertical space

%    \includegraphics[width=4cm]{example-image-a} % Add an image before the title. you will need to replace the image path with your own

%    \vspace{0.5cm} % Add vertical space before title

    \textbf{\fontsize{18}{36}\selectfont \thetitle} % Font Size and Bold Title

     \vspace{0.05cm} % Add vertical space before subtitle
%    \textit{\Large \theauthor}  % Subtitle / Author
    \vspace{\baselineskip} % Add vertical space after subtitle
     \rule{\textwidth}{0.4pt} % Add a horizontal line

   \end{leftmark}
%    \thispagestyle{empty} % Prevent header/footer on the title page
}


% Section Formatting
\titleformat{\section}
  {\normalfont\fontsize{18}{22}\bfseries} % Font and style
  {\thesection}         % Section number
  {1em}                   % Horizontal space after section number
  {}                     % Code before the section name
  []                     % Code after the section name

\titleformat{\subsection}
  {\normalfont\fontsize{14}{18}\bfseries} % Font and style
  {\thesubsection}         % Subsection number
  {1em}                   % Horizontal space after subsection number
  {}                     % Code before the subsection name
  []                     % Code after the subsection name

\setlength{\parindent}{0pt}

\title{Computing platforms (Spring 2025)\newline
week 6}
\author{Juha-Pekka Heikkilä}



\pagestyle{fancy}
\fancyhf{}

\renewcommand{\headrulewidth}{0pt}

\newcommand{\footerline}{\makebox[\textwidth]{\hrulefill}}

\newcommand{\footercontent}{%
    \begin{tabular}{@{}l@{}}
        \footerline \\
        \leftmark \hfill \rlap{\thepage}
    \end{tabular}
}

\fancyfoot[C]{\footercontent}


\newcommand{\exercise}[1]{
    \section*{Tehtävä #1}
    \markboth{Tehtävä #1}{}
}

\addtolength{\hoffset}{-1.75cm}
\addtolength{\textwidth}{3.5cm}
%\addtolength{\voffset}{-3cm}
%\addtolength{\textheight}{6cm}
%\setlength{\parindent}{0pt}



% (a), (b), (c)
\newlist{kohta}{enumerate}{1}
\setlist[kohta,1]{
  label=\textbf{\makebox[1cm][l]{\Huge\text{(\stylishfont\alph*)}}},
  leftmargin=!,
  labelindent=0pt
}

% (1), (2), (3)
\newlist{alakohta}{enumerate}{1}
\setlist[alakohta,1]{
  label=\textbf{\makebox[1cm][l]{\Large\text{(\arabic*)}}},
  leftmargin=!,
  labelindent=0pt
}

% termi: selitys
\newlist{kuvaus}{description}{1}
\setlist[kuvaus]{%
  font=\bfseries\stylishfont,
  labelsep=0.5cm,
  leftmargin=2.5cm,
  style=nextline
}

\newcommand{\korostus}[2][yellow]{\colorbox{#1}{\strut #2}}
%\korostus{Yksi kirjoittaja on jo sisällä}
%\korostus[red]{Lukijan täytyy odottaa jos kirjoittajia on paikalla}
%\korostus[orange]{Tämä osa ei ole suoritettavissa}


\newcommand{\evalslantti}[4][-12]{%
%  \left. #2 \,\right|% ei indeksejä tähän
  \mkern-10mu\raisebox{0pt}[0pt][0pt]{\rotatebox{#1}{$\Big|$}}% vinoviiva päälle
  \mkern3mu{}_{\!#3}^{\!#4}% arvot viivan oikealle puolelle
}


\newcommand{\evalraise}{1.2ex}
\newcommand{\evallow}{1.2ex}

% vino eval-viiva, arvot oikealla (oletus: -12)
% \evalslant[asteet]{lauseke}{ala}{yla}
\newcommand{\evalslant}[4][-12]{%
  \left. #2 \,\right.%
  \mkern-10mu\raisebox{0pt}[0pt][0pt]{\rotatebox{#1}{$\Big|$}}%
  \mkern2mu{}^{\raisebox{\evalraise}{$\scriptstyle #4$}}_{\raisebox{-\evallow}{$\scriptstyle #3$}}%
}



% vino eval-viiva ENNEN lauseketta
% \evalslantpre[asteet]{lauseke}{ala}{yla}
\newcommand{\evalslantpre}[4][-12]{%
  % viiva ja rajat
  \raisebox{0pt}[0pt][0pt]{\rotatebox{#1}{$\Big|$}}%
  \mkern2mu{}^{\raisebox{\evalraise}{$\scriptstyle #4$}}_{\raisebox{-\evallow}{$\scriptstyle #3$}}%
  % itse lauseke
  \mkern4mu\left. #2 \right.%
}


\DeclareMathOperator{\Var}{Var}
\DeclareMathOperator{\Cov}{Cov}
\DeclareMathOperator{\Corr}{Corr}
\usepackage{tikz}
\usetikzlibrary{automata, positioning, arrows.meta}
\usepackage{amssymb,amsmath,graphicx,color}
\usepackage{forest}
\usepackage{booktabs}
\usepackage{siunitx}
\usepackage{amsthm}
\newtheorem{lause}{Lause}
\renewcommand{\proofname}{Todistus}
\renewcommand{\qedsymbol}{$\blacksquare$}

\newcommand{\set}[1]{\left\{\,#1\,\right\}}
\newcommand{\abs}[1]{\lvert#1\rvert}

\newcommand{\N}{\mathbb{N}}
\newcommand{\Pot}{{\cal P}}

\newcommand{\rma}{\mathrm{a}}
\newcommand{\rmb}{\mathrm{b}}
\newcommand{\rmc}{\mathrm{c}}
\newcommand{\code}[1]{\left\langle\,#1\,\right\rangle}


\title{TKT20005 Laskennan mallit Viikko5}
\date{}

\begin{document}

\maketitle

\exercise{1. Turingin kone.} Tässä tehtävässä harjoitellaan lukemaan Turingin koneen tilakaaviota.
  
Esitä allaolevan Turingin koneen laskenta syötteellä 0111
luettelemalla laskennan aikana syntyvät tilanteet, kuten
luentomateriaalin sivulla~181 (tai Sipserin sivun~172 alaosassa).
\bigskip

\begin{tabular}{rrrrrrrrr} \toprule
$q_0 0111$&  \vdash &  \korostus{q0} 0 &               1    &               1 &               1 & &\\
&  \vdash &               a & \korostus{q1} 1    &               1 &               1 & &\\
&  \vdash &               a &               1    &               1 &               1 & \korostus{q1} &\\
&  \vdash &               a &               1    &               1 & \korostus{q2} 1 & &\\
&  \vdash &               a & \korostus{q3} 1    &               1 &               B & &\\
&  \vdash & \korostus{q4} a &               1    &               1 &               B & &\\
&  \vdash &               a & \korostus{q0} 1    &               1 &               B & &\\
&  \vdash &               a &               b    &               1 & \korostus{q1} B & &\\
&  \vdash &               a &               b    & \korostus{q2} 1 &               B & &\\
&  \vdash &               a & \korostus{q3} b    &               B &               B & &\\
&  \vdash &               a &               b    & \korostus{q5} B &               B & &\\
&  \vdash &               a &               b    &               B &               B & \korostus{q5} &\\
&  \vdash &               a &               b    &               B & \korostus{q6} B & &\\
&  \vdash &               a &               b    & \korostus{q9} B &               X & &\\
&  \vdash &               a & \korostus{q9} b    &               B &               X & &\\
&  \vdash &               a &               X    & \korostus{q10}B &               X & & \\
&  \vdash &               a &               X    &               B &               X & \korostus{q10} & \rightarrow \, $q_{acc}$\\
\bottomrule
\end{tabular}





\pagebreak
\exercise{2. Laskentaongelman ratkaisu Turingin koneella.}
Turingin konetta voi luonnollisella tavalla käyttää paitsi kielen
tunnistamiseen myös funktion laskemiseen.
Määritellään, että Turingin kone laskee funktion
$f\colon\Sigma^\ast\rightarrow\Sigma^\ast$, jos millä
tahansa syötemerkkijonolla $w$ se pysähtyy tilanteeseen, jossa
nauhalla on alussa merkkijono $f(w)$ ja sen jälkeen pelkkiä
tyhjämerkkejä.

Tarkastelemme seuraavassa binäärikoodattujen luonnollisten lukujen
käsittelemistä Turingin koneella.
Kun $n\in\N$ on luonnollinen luku, käytämme merkintää
$\code{n}$ tarkoittamaan luvun $n$ binääriesitystä
(aakkoston $\Sigma=\set{0,1}$ merkkijonona) ilman
turhia etunollia.
Esim.\ $\code{12}=1100$.

Määritellään lukujen binääriesityksille edeltäjäfunktio
$p$ niin, että jos $w=\code{n+1}$ jollain $n\in\N$, niin
$p(w)=\code{n}$.
Jos $w$ ei ole muotoa $\code{n+1}$ (eli $w=\varepsilon$
tai $w=\code{0}$ tai $w$:ssä
on ylimääräisiä etunollia), määritellään
$p(w)=\varepsilon$.

Kuvaile yksinauhainen Turingin kone, joka laskee funktion $p$.
Sopiva tarkkuus esitykselle on suunnilleen vastaava kuin
luentomateriaalin sivulla~186.
Toisin sanoen selitä, miten nauhaa käytetään ja muut mahdolliset
korkean tason periaatteet, mutta älä mene yksittäisten tilojen ja
tilasiirtymien tasolle.\\
\newline






\medskip
Kone toimii kolmessa vaiheessa:

\medskip
\begin{alakohta}

\item{\bf Kelpoisuustesti ja tyhjätapaukset:}
\begin{itemize}
  \item Jos ensimmäinen solu on tyhjä (eli $w=\varepsilon$): pysähdy koska nauha on valmiiksi tyhjä
  \item Jos ensimmäinen merkki on $0$:
    \begin{itemize}
      \item Jos seuraava solu on tyhjä (eli $w=0$): pyyhi tuo $0$ ja pysähdy (tulos $\varepsilon$)
      \item Muuten (pituus $\ge 2$ ja johtava $0$ eli ylimääräisiä etunollia): pyyhi koko syöte (korvaa kaikki 0/1:t tyhjällä) ja pysähdy, tulos $\varepsilon$
    \end{itemize}
  \item Muuten ensimmäinen merkki on $1$ (kelvollinen koodi $\mathrm{bin}(n{+}1)$): jatka vaiheeseen 2 
\end{itemize}

\item{\bf Vähennä yksi (binääri -1 oikealta):}\\
Siirry syötteen loppuun (kulje oikealle kunnes tyhjö, sitten yksi vasemmalle)
\begin{itemize}
  \item Jos solu on $1$: kirjoita $0$ ja siirry vaiheeseen 3
  \item Jos solu on $0$: toista, kunnes löydät $1$:n vasemmalta:
    \begin{itemize}
      \item jokaiselle peräkkäiselle $0$:lle: kirjoita $1$ ja siirry yhden askeleen vasemmalle;
      \item ensimmäisen $1$:n kohdalla: kirjoita $0$ ja siirry vaiheeseen 3
    \end{itemize}
\end{itemize}
Tässä vaiheessa nauhalla on $\mathrm{bin}(n)$, mutta
tapauksessa $w=10^k$ vasemmalle jäi yksi johtava nolla

\pagebreak
\item{\bf Poista johtava nolla tarvittaessa:}\\
Palaa vasempaan reunaan (kulje vasemmalle tyhjään, sitten yksi oikealle)
\begin{itemize}
  \item Jos vasemmanpuoleisin merkki on $1$: pysähdy.
  \item Jos vasemmanpuoleisin merkki on $0$:
    \begin{itemize}
      \item jos seuraava solu on tyhjä (jäännös "0", tapaus $w=1$): pysähdy ja jätä "0" paikalleen
      \item muuten (pituus $\ge 2$ ja johtava 0): poista vasemman laidan 0 siirtämällä sanaa yhden solun verran vasemmalle:
        \begin{itemize}
          \item kulje oikealle ja jokaisessa solussa kopioi sen sisältö edelliseen soluun (ylikirjoita); jatka kunnes kohtaat tyhjän;
          \item mene yksi vasemmalle ja kirjoita tyhjä (tyhjää viimeinen kopio)
          \item palaa vasempaan reunaan ja pysähdy
        \end{itemize}
    \end{itemize}
\end{itemize}
\end{alakohta}

\medskip
\noindent{\bf testejä}
\begin{itemize}
  \item $w=\varepsilon \;\Rightarrow\; \varepsilon$ (Vaihe1)
  \item $w=0 \;\Rightarrow\; \varepsilon$ (Vaihe 1)
  \item $w=1 \;\Rightarrow\; 0$ (Vaihe 2 tuottaa ”0”; Vaihe 3 tunnistaa yksittäisen nollan)
  \item $w=10 \;\Rightarrow\; 01 \stackrel{\text{V3}}{\Rightarrow} 1$
  \item $w=101000 \;\Rightarrow\; 100111$ (3 ei poista mitään)
\end{itemize}





\pagebreak
\exercise{3. Turing-tunnistettavuus.}Tässä tehtävässä tarkastellaan, mitä Turing-tunnistettavuus tarkoittaa, ja harjoitellaan, kuinka jokin kieli todistetaan Turing-tunnistettavaksi.

    Tarkastellaan kieltä
\[A=\set{\code{M,w,q}\mid
\mbox{$M$ on Turingin kone, joka syötteellä $w$ menee ainakin kerran
tilaan $q$}}.\]
Todista, että kieli $A$ on Turing-tunnistettava.

\bigskip


    
\begin{proof}
Määritellään Turingin kone R

\medskip
\noindent Syöte: koodattu kolmikko $\langle M,w,q\rangle$

\begin{enumerate}
\item Tarkista, että $q$ on $M$:n tilajoukon koodaus. Jos ei ole hylkää
\item Simuloi $M$:ää syötteellä $w$ askel askeleelta universaalilla simulaattorilla
      Pidä kirjaa $M$:n nykyisestä tilasta.
      \begin{itemize}
        \item Jos simulaation jossain askeleessa $M$:n tila on $q$, hyväksy
        \item Jos $M$ pysähtyy hyväksyen tai hyläten ennen kuin tila $q$ esiintyy, hylkää.
        \item Muussa tapauksessa  $M$ jatkaa ikuisesti eikä koskaan vieraile $q$:ssa.
      \end{itemize}
\end{enumerate}

\noindent  
Jos $\langle M,w,q\rangle\in A$, niin simuloinnissa on jokin hetki,
jolloin $M$:n tila on $q$, ja $R$ hyväksyy\\

Jos $\langle M,w,q\rangle\notin A$, on kaksi tapausta: joko $M$ 
pysähtyy koskaan vierailematta $q$:ssa (eli $R$ hylkää), 
tai $M$ ei pysähdy eikä vieraile $q$:ssa, jolloin $R$ jatkaa 
simulaatiota ikuisesti (ei hyväksy). Eli $R$ hyväksyy 
täsmälleen $A$:n syötteet, joten $A$ on Turing-tunnistettava.
\end{proof}





\pagebreak
\exercise{4. Turing-ratkeavuus.}
Tässä tehtävässä tarkastellaan, mitä Turing-ratkeavuus tarkoittaa, ja harjoitellaan, kuinka jokin kieli todistetaan Turing-ratkeavaksi.\\

Kieli $T$ koostuu kaikista sellaisista pareista $\code{M,w}$,
joissa $M$ on Turingin kone, $w$ jokin sen syöte
ja nämä toteuttavat ehdon, että koneen $M$ laskennassa
syötteellä $w$ ainakin yksi tila toistuu ainakin kaksi kertaa.\\

Todista, että kieli on ratkeava.\\

\bigskip





%Olkoon
\[
T=\{\langle M,w\rangle \mid \text{$M$:llä syötteellä $w$ jokin ohjaustila esiintyy vähintään kahdesti}\}
\]
\\


$T$ on Turing-ratkeava \\

\begin{proof}
Rakennetaan päättävä kone $D$ syötteelle $\langle M,w\rangle$

Laske $N:=|Q_M|$, siis tilojen lukumäärä $M$:n kuvauksesta\\
Simuloi $M$:ää syötteellä $w$
enintään $N$ siirtymää pitäen kirjaa ohjaustiloista missä käyty

\begin{itemize}
\item Jos simulaation aikana jokin tila toistuu, hyväksytään
\item Jos $M$ pysähtyy ennen toistoa, hylätään
\item Simulaatio on rajattu $N$ askeleeseen jolloin $D$ pysähtyy aina
\end{itemize}

jos jokin tila toistuu jossain vaiheessa, kyyhkyslakkaperiaate
takaa, että toisto tapahtuu viimeistään $N$ askeleen kuluessa 
(koska $N+1$ tilakäyntiä $N$:ssä tilassa pakottaa kahden käynnin 
osumaan samaan tilaan). \\

\rightarrow \, Siksi $D$ havaitsee toiston ja hyväksyy.
Jos toistoa ei ole, silloin $M$ pysähtyy ilman toistoa 
(hylkäys yllä), tai muuten $N$ askelta myöhemminkään toistoa 
ei esiinny — mikä on mahdotonta, joten tätä tapausta ei synny. 
Siis $D$ päättää kielen $T$
\end{proof}



\end{document}