\documentclass[12pt,a4paper]{article}
% !TEX program = xelatex
\usepackage[utf8]{inputenc}
\usepackage[T1]{fontenc}
\usepackage[finnish]{babel}
\usepackage[utf8]{inputenc}
\usepackage{graphicx}
\usepackage{titling}
\usepackage{titlesec}
\usepackage{booktabs}
\usepackage{fancyhdr}
\usepackage{lipsum}
\usepackage{comment, mdframed}
\usepackage{enumitem}
\usepackage{xcolor}
\usepackage{longtable}
%\usepackage{cite}
\usepackage{pgfgantt}
\usepackage{amsmath, amssymb}
\usepackage{tikz}
\usepackage[margin=1in]{geometry}
\usepackage[backend=biber, style=numeric]{biblatex}
%\usepackage{hyperref}
\usepackage{bookmark}
\usepackage{enumitem}
\usepackage{amsmath}
\usepackage{listings}
\lstset{language=Python, basicstyle=\ttfamily\small, breaklines=true,columns=fullflexible}
\lstset{escapeinside={(*@}{@*)}}
\usepackage{fontspec}
\setmainfont{Arial}
\newfontfamily\stylishfont{Noteworthy}
%\newfontfamily\stylishfont{Zapfino}
%\addbibresource{references.bib}
\usetikzlibrary{calc}
\usepackage{xcolor}

\lstdefinestyle{pidstyle}{
    basicstyle=\ttfamily\footnotesize,
    breaklines=true,
    escapechar=\#, % Define escape character for inline LaTeX commands
    linewidth=\textwidth,
    basicstyle=\ttfamily\scriptsize
}

\renewcommand{\maketitle}{%
  \begin{leftmark}
    \vspace*{\baselineskip} % Add a bit of vertical space

%    \includegraphics[width=4cm]{example-image-a} % Add an image before the title. you will need to replace the image path with your own

%    \vspace{0.5cm} % Add vertical space before title

    \textbf{\fontsize{18}{36}\selectfont \thetitle} % Font Size and Bold Title

     \vspace{0.05cm} % Add vertical space before subtitle
%    \textit{\Large \theauthor}  % Subtitle / Author
    \vspace{\baselineskip} % Add vertical space after subtitle
     \rule{\textwidth}{0.4pt} % Add a horizontal line

   \end{leftmark}
%    \thispagestyle{empty} % Prevent header/footer on the title page
}


% Section Formatting
\titleformat{\section}
  {\normalfont\fontsize{18}{22}\bfseries} % Font and style
  {\thesection}         % Section number
  {1em}                   % Horizontal space after section number
  {}                     % Code before the section name
  []                     % Code after the section name

\titleformat{\subsection}
  {\normalfont\fontsize{14}{18}\bfseries} % Font and style
  {\thesubsection}         % Subsection number
  {1em}                   % Horizontal space after subsection number
  {}                     % Code before the subsection name
  []                     % Code after the subsection name

\setlength{\parindent}{0pt}

\title{Computing platforms (Spring 2025)\newline
week 6}
\author{Juha-Pekka Heikkilä}



\pagestyle{fancy}
\fancyhf{}

\renewcommand{\headrulewidth}{0pt}

\newcommand{\footerline}{\makebox[\textwidth]{\hrulefill}}

\newcommand{\footercontent}{%
    \begin{tabular}{@{}l@{}}
        \footerline \\
        \leftmark \hfill \rlap{\thepage}
    \end{tabular}
}

\fancyfoot[C]{\footercontent}


\newcommand{\exercise}[1]{
    \section*{Tehtävä #1}
    \markboth{Tehtävä #1}{}
}

\addtolength{\hoffset}{-1.75cm}
\addtolength{\textwidth}{3.5cm}
%\addtolength{\voffset}{-3cm}
%\addtolength{\textheight}{6cm}
%\setlength{\parindent}{0pt}



% (a), (b), (c)
\newlist{kohta}{enumerate}{1}
\setlist[kohta,1]{
  label=\textbf{\makebox[1cm][l]{\Huge\text{(\stylishfont\alph*)}}},
  leftmargin=!,
  labelindent=0pt
}

% (1), (2), (3)
\newlist{alakohta}{enumerate}{1}
\setlist[alakohta,1]{
  label=\textbf{\makebox[1cm][l]{\Large\text{(\arabic*)}}},
  leftmargin=!,
  labelindent=0pt
}

% termi: selitys
\newlist{kuvaus}{description}{1}
\setlist[kuvaus]{%
  font=\bfseries\stylishfont,
  labelsep=0.5cm,
  leftmargin=2.5cm,
  style=nextline
}

\newcommand{\korostus}[2][yellow]{\colorbox{#1}{\strut #2}}
%\korostus{Yksi kirjoittaja on jo sisällä}
%\korostus[red]{Lukijan täytyy odottaa jos kirjoittajia on paikalla}
%\korostus[orange]{Tämä osa ei ole suoritettavissa}


\newcommand{\evalslantti}[4][-12]{%
%  \left. #2 \,\right|% ei indeksejä tähän
  \mkern-10mu\raisebox{0pt}[0pt][0pt]{\rotatebox{#1}{$\Big|$}}% vinoviiva päälle
  \mkern3mu{}_{\!#3}^{\!#4}% arvot viivan oikealle puolelle
}


\newcommand{\evalraise}{1.2ex}
\newcommand{\evallow}{1.2ex}

% vino eval-viiva, arvot oikealla (oletus: -12)
% \evalslant[asteet]{lauseke}{ala}{yla}
\newcommand{\evalslant}[4][-12]{%
  \left. #2 \,\right.%
  \mkern-10mu\raisebox{0pt}[0pt][0pt]{\rotatebox{#1}{$\Big|$}}%
  \mkern2mu{}^{\raisebox{\evalraise}{$\scriptstyle #4$}}_{\raisebox{-\evallow}{$\scriptstyle #3$}}%
}



% vino eval-viiva ENNEN lauseketta
% \evalslantpre[asteet]{lauseke}{ala}{yla}
\newcommand{\evalslantpre}[4][-12]{%
  % viiva ja rajat
  \raisebox{0pt}[0pt][0pt]{\rotatebox{#1}{$\Big|$}}%
  \mkern2mu{}^{\raisebox{\evalraise}{$\scriptstyle #4$}}_{\raisebox{-\evallow}{$\scriptstyle #3$}}%
  % itse lauseke
  \mkern4mu\left. #2 \right.%
}


\DeclareMathOperator{\Var}{Var}
\DeclareMathOperator{\Cov}{Cov}
\DeclareMathOperator{\Corr}{Corr}

\newcommand{\set}[1]{\left\{\,#1\,\right\}}

\newcommand{\N}{\mathbb{N}}

\newcommand{\mot}{$\Box$}


\title{TKT20005 Laskennan mallit Viikko1}
\date{}

\begin{document}

\maketitle

\exercise{1 Implikaatio ja ekvivalenssi.}
\begin{enumerate}
  \item Alicella ja Bobilla on molemmilla lehmiä, jotka laiduntavat samalla niityllä. Tiedämme, että seuraavat loogiset lauseet niityllä olevista lehmistä ovat totta:
    \begin{align*}
      &\textrm{``Lehmä on Alicen.''} \implies \textrm{``Lehmä on ruskea.''}\\
      &\textrm{``Lehmä on Bobin.''} \iff \textrm{``Lehmällä on sarvet.''}
    \end{align*}
    Luonnollisella kielellä yllä olevat lauseet voidaan ilmaista esim.:
    \begin{itemize}
        \item Jos lehmä on Alicen, se on ruskea.
        \item Lehmä on Bobin, jos ja vain jos sillä on sarvet.
    \end{itemize}
\end{enumerate}

\begin{kohta}
  \item % a
    \begin{alakohta}
      \item \textbf{Lehmä on ruskea.} Implikaatiosta "Jos lehmä on Alicen, 
      se on ruskea" ($A \Rightarrow R$) ei voi päätellä mitään omistajasta 
      jos lehmä on ruskea. Ruskea lehmä voi olla Alicen, mutta se voi olla 
      myös jonkun muun lehmä. 
      \textbf{Omistajasta ei siis voida päätellä mitään.}

      \item \textbf{Lehmä ei ole ruskea.} Implikaatiosta $A \Rightarrow R$ 
      seuraa loogisesti $\neg R \implies \neg A$. Tämä 
      tarkoittaa: "Jos lehmä ei ole ruskea, se ei ole Alicen." Koska 
      lehmä ei ole ruskea,
      \textbf{voimme varmuudella sanoa, että lehmä ei ole Alicen}

      \item \textbf{Lehmällä on sarvet.} Ekvivalenssi "Lehmä on Bobin, jos
      ja vain jos sillä on sarvet" ($B \iff S$) tarkoittaa, että lauseilla
      on aina sama arvo.
      \textbf{Koska lehmällä on sarvet, sen on oltava Bobin.}

      \item \textbf{Lehmällä ei ole sarvia.} Ekvivalenssin ($B \iff S$)
      mukaan, jos lehmällä ei ole sarvia, se ei voi olla Bobin.
      \textbf{Lehmä ei ole Bobin.}
    \end{alakohta}

  \item % b
    \begin{alakohta}
      \item Jos $D$ on tosi, implikaatiosta $A \Rightarrow D$ 
      \textbf{ei voida päätellä mitään $A$:n totuusarvosta.}

      \item Jos $D$ ei ole tosi, niin $\neg D \Rightarrow \neg A$ nojalla 
      voidaan todeta, että \textbf{A ei ole tosi.}

      \item Jos tiedetään että $E$ on tosi, ekvivalenssin $B \iff E$ nojalla myös
      \textbf{B on tosi.}

      \item Jos tiedetään että $E$ ei ole tosi, ekvivalenssin $B \iff E$ nojalla myös
      \textbf{B ei ole tosi.}
    \end{alakohta}
\end{kohta}








\pagebreak
\exercise{2 Vastaesimerkki ja epäsuora todistus.}
  Tunnetusti kahden luonnollisen luvun summa on luonnollinen luku. Toisin sanoen pätee:
  \[
    a\in \N \textrm{ ja } b\in \N \implies a + b\in\N
  \]
  \begin{enumerate}
  \item Tiedetään, että $a$ on luonnollinen luku ja $b$ ei ole luonnollinen luku. Voidaanko tästä päätellä, että $a + b$ ei ole luonnollinen luku? Perustele vastauksesi täsmällisesti antamalla vastaesimerkki tai todistus perustuen yllä olevaan implikaatioon ja epäsuoraan todistustekniikkaan.
  \item Tiedetään, että $a$ on luonnollinen luku ja $a + b$ ei ole luonnollinen luku. Voidaanko tästä päätellä, että $b$ ei ole luonnollinen luku? Perustele vastauksesi täsmällisesti antamalla vastaesimerkki tai todistus perustuen yllä olevaan implikaatioon ja epäsuoraan todistustekniikkaan.
  \end{enumerate}



  \begin{alakohta}
  \item % a
    Väite on: $a\in\N \textrm{ ja } b\notin\N \Rightarrow a+b\notin\N$

    On löydettävä sellaiset luvut $a$ ja $b$, joilla alkuoletus pätee
    ($a\in\N$ ja $b\notin\N$), mutta johtopäätös ei päde ($a+b\in\N$)

    Valitaan a=3 ja b=-2
    \begin{itemize}
      \item Tällöin $a=3$ on luonnollinen luku.
      \item Luku $b=-2$ ei ole luonnollinen luku.
      \item a+b = 3 + (-2) = 1. Luku 1 on luonnollinen luku; siis a+b$\in\N$
    \end{itemize}
    Koska löysimme tapauksen, missä oletukset on voimassa mutta väite
    ei pidä paikkaansa, emme voi yleisesti sanoa $a+b$ ei ole
    luonnollinen luku. \textbf{Väite on siis epätosi.} 

  \item % b
    Väite on: $a\in\N \textrm{ ja } a+b\notin\N \Rightarrow b\notin\N$

    \textbf{Väite on tosi.} Todistetaan tämä epäsuorasti.

    Oletetaan, että $a\in\N$ ja $a+b\notin\N$. Oletetaan vastoin väitettä,
    että $b$ on luonnollinen luku: $b\in\N$.

    Nyt meillä on tiedossa kaksi asiaa:
    \begin{itemize}
      \item Alkuperäisen oletuksen mukaan $a\in\N$.
      \item Vastaoletuksen mukaan $b\in\N$.
    \end{itemize}
    Tehtävänannossa annetun perustiedon mukaan kahden luonnollisen luvun
    summa on aina luonnollinen luku. Koska a ja b ovat molemmat
    oletustemme mukaan luonnollisia lukuja, niiden summan a+b on siis
    pakko olla luonnollinen luku, eli $a+b\in\N$

    Tämä on kuitenkin ristiriidassa alkuperäisen oletuksen kanssa,
    jonka mukaan $a+b\notin\N$ eli alkuperäinen \textbf{väite on tosi.} 
\end{alakohta}
\end{document}